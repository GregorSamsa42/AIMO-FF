\documentclass[oneside]{memoir}
\usepackage[utf8]{inputenc}
\usepackage[ngerman]{babel}
\usepackage[T1]{fontenc}
\usepackage{lettrine}
\DisemulatePackage{setspace}
\usepackage{setspace}
\onehalfspacing

\makeatletter
\newenvironment{chapquote}[2][2em]
  {\setlength{\@tempdima}{#1}%
   \def\chapquote@author{#2}%
   \parshape 1 \@tempdima \dimexpr\textwidth-2\@tempdima\relax%
   \itshape}
  {\par\normalfont\hfill--\ \chapquote@author\hspace*{\@tempdima}\par\bigskip}
\makeatother
\pretitle{\begin{center}\Huge\bfseries}
\title{In Jürgen's Keller}
\posttitle{\par\vskip1em{\normalfont\normalsize\scshape Eine AIMO-Geschichte\par\vfill}\end{center}}
\author{Herausgegeben von einer nichtleeren Teilmenge der AIMO-Teilnehmer 2020}
\predate{\vfill\begin{center}\large}
\begin{document}
\begin{titlingpage}
\maketitle
\end{titlingpage}
\chapter{Prolog in der Hölle}

\begin{chapquote}{Fyodor Dostoevsky}
\glqq Aus hundert Kaninchen wird niemals ein Pferd und aus hundert Verdachtsgründen niemals ein Beweis.\grqq
\end{chapquote}

\lettrine{M}{arc Hieke} schaute müde von den zahlreichen Mathematikaufgaben vor ihm wieder auf.

\chapter{Definitionen und Grundbegriffe} %bis 1.5
\begin{chapquote}{Veran Stojanović}
\glqq Was wir wissen, ein Epsilon. Was wir nicht wissen, ein 1/Epsilon.\grqq
\end{chapquote}
\textit{Der 20. April 2020} \\ 
\lettrine{D}{ie Teilnehmerliste} auf der rechten Seite von Georgis Bildschirm füllte sich langsam, als Jürgen Prestin einen AIMO-Teilnehmer nach dem anderen in die Zoomsitzung ließ, in der er die am Abend davor abgeschickten Aufgaben zu besprechen gedachte. Die ihm inzwischen vertrauten Namen purzelten geradezu in die Sitzung, schon drei Minuten nach 16 Uhr waren so gut wie alle anwesend. Georgi nahm einen Schluck von der Tüte Orangensaft neben seinem Laptop und setzte seine Kopfhörer auf. Veran, Christian und Patrick waren bereits auf einem separaten Discordserver, in dem sie für gewöhnlich während den Seminaren ratschten – er beeilte sich und trat diesem bei. \\
„Und nach dem Eisensteinkriterium ist P irreduzibel, glaube ich“, beendete Christian seinen Gedanken, worauf Veran zögernd sein Einverständnis gab. \\
„Hi“, sagte Georgi und erhielt einen Gruß von den beiden anderen. „Wo ist Patrick?“ \\
„Unklar“, antwortete Veran schroff. „Wahrscheinlich spielt er gerade GTA oder so.“
Ein Rauschen war von Patricks Mikrofon zu hören. Vermutlich hatte er sich gerade hingesetzt. \\ 
„Jungs, ich hab‘ mir grad so dick n Salat geholt“, gluckste Patrick vergnügt. „Meine Mutter hat wieder welche gekauft.“
Georgi wollte etwas dazu sagen, als sich Jürgen Prestin meldete. Er werde nun anfangen, ohne auf die Nachzügler zu warten. Es fehlte nur noch Marc, der schon einige Seminare vor diesem hatte sausen lassen. Keiner der vieren wusste, ob er sich entschuldigt hatte oder warum genau er nicht erschien. Der Dozent fragte, wer denn zuerst eine Aufgabe vorstellen möchte, und während Georgi gerade darüber nachdachte, ob er die zwei Aufgaben, die er vorbereitet hatte, jetzt oder später vorstellen wollte, klinkte sich Christian aus dem Discord aus, hob auf Zoom seine Stummschaltung auf und fing an, über die 5 zu referieren. 
Nachdem er fertig war, trat er dem Channel mit den anderen wieder bei.  \\
„Voll kreativ“, lobte ihn Veran, worauf Christian sich bedankte. Er fragte daraufhin die anderen beiden im Call, ob sie die Aufgabe ähnlich gelöst hatten. \\
„Bruder, ich hab‘ grad gar nicht aufgepasst“, erwiderte Patrick leicht aufgebracht. „Es hat mich wieder dieser komische Typ angerufen.“ \\
„Der Nummernachbar?“, fragte Georgi. Der Unbekannte mit derselben Nummer wie Patrick bis auf die letzte Ziffer hatte ihn bereits vor dem ersten Bad Homburg-Seminar kontaktiert und hatte nur wirres Zeug von sich gegeben. Obwohl Sofia und die anderen sich über Michael, so hatte er sich genannt, lustig machten, war er Patrick ganz und gar nicht geheuer. \\
„Ja“, antwortete Patrick. „Er war aufgeregt und hat irgendwas davon geredet, dass ich vorsichtig sein sollte. Ich hab‘ einfach aufgelegt, ich fand das zu seltsam.“ \\
„Fishy“, sagte Christian. „Ich glaube, du brauchst das nicht allzu ernst nehmen.“ \\
„Ja, glaub ich auch nicht.“, meinte Patrick und richtete seine Aufmerksamkeit auf seinen Salat. „Der Typ ist sowieso nicht ganz klar im Kopf.“

\bigskip \bigskip \bigskip \bigskip 

%1.2

\bigskip \bigskip \bigskip \bigskip

\textit{Der 21. April 2020} \\ 
„Der Nummernnachbar von Patrick hat sich übrigens wieder gemeldet.“ Georgi telefonierte gerade mit seinem Freund Maximilian Keßler, welcher in den zwei Jahren zuvor an der AIMO teilgenommen und es 2019 auch zur IMO geschafft hatte, und berichtete ihm eifrig über die neuesten Geschehnisse. \\
 „Oh echt? Ich dachte der hat längst aufgehört, Patrick zu schreiben?“, wunderte sich Max. \\
 „Ja, das letzte Mal haben wir in Bad Homburg von ihm gehört. Aber gestern während dem AIMO-Meeting hat er wieder bei Patrick angerufen...“ Georgi erzählte ihm also alles über den mysteriösen Anruf. \\
 „Eure AIMO ist echt witzig, so etwas ist letztes und vorletztes Jahr nicht passiert!“, meinte Max Keßler daraufhin amüsiert. \\
 „Ja, das stimmt“, fand auch Georgi, „Aber dafür konntest du immerhin zu den Seminaren in Reallife fahren... \\
  „Oh ja, vor allem die Woche in Oberwolfach hätte ich dir auch gegönnt!“ \\
   „Jaa, Corona ist echt so eine Bitch.“ \\
    „Ich sag mal so, is nicht schön, aber wat willste machen?“, scherzte Max daraufhin und beide schmunzelten. Georgi erzählte weiter: \\
    „Aber mal im Ernst, die Zoom-Meetings sind einfach nicht das gleiche Erlebnis. Ich würde mich so langweilen, wenn ich nicht währenddessen mit den anderen über Discord reden würde. Marc zum Beispiel kommt seit einigen Tagen gar nicht mehr zu den Seminaren...“ \\
    „Marc Hieke?“, fragte Max nach. \\
     „Ja“, bestätigte Georgi. \\
     „Fishy. Gerade der sollte sich doch mal bisschen mehr anstrengen. Seine Platzierung auf der Schätzliste sieht im Moment gar nicht gut aus“, bemerkte Max. Georgi stimmte ihm zu, und nach einigen weiteren Diskussionen über die Schätzliste und die Seminare klopfte es an Georgis Tür und man hörte seine Mutter rufen: „Essen ist fertig!“ \\
     „Ah, ich muss dann wohl gehen“, sagte Georgi zu Max. \\
     „Dann guten Appetit!“, antwortete Max und die beiden verabschiedeten sich.
     
     \bigskip \bigskip \bigskip \bigskip
     %1.5
     \textit{Der 24. April 2020} \\ 
     
     \bigskip \bigskip \bigskip \bigskip
     %1.3
     \textit{Der 25. April 2020} \\ 

\chapter{Einleitende Überlegungen} %bis 1.9
\begin{chapquote}{Sören Kierkegaard}
\glqq Wie ist doch die ganze Natur so ominös! Voll Vorbedeutung ist mir der Vögel Flug, ihr Schrei, der Fische ausgelaßnes Schlagen gegen die Oberfläche des Wassers, ihr Verschwinden in der Tiefe, fernes Hundegebell, eines Wagens fernes Gerassel, Schritte, die von fernher widerhallen. Nicht sehe ich Gespenster in dieser nächtlichen Stunde, nicht sehe ich, was war, sondern das, was kommen wird.\grqq
\end{chapquote}

     
       \bigskip \bigskip \bigskip \bigskip
     %1.6
     \textit{Der 26. April 2020} \\ 
     
       \bigskip \bigskip \bigskip \bigskip
     %1.7
     \textit{Der 30. April 2020} \\ 
     
       \bigskip \bigskip \bigskip \bigskip
     %1.8
     \textit{Der 2. Mai 2020} \\ 
     
       \bigskip \bigskip \bigskip \bigskip
     %1.9
     \textit{Der 3. Mai 2020} \\ 

\newpage

Willkommen bei der AIMO, \\
euch hier zu haben sind wir froh. \\
Bevor wir starten mit den Termen, \\
müsst ihr uns erst kennenlernen. \\  \\

Ich bin Käpt'n dieser Bande \\
Zu ganz vielem wohl im Stande. \\
Mit viel Wumms werd' ich erziehen, \\
sonst wär' ich nicht der Herr Prestin. \\

Mit stummen Blicken folge ich \\
Jedem Schritt, ganz unheimlich. \\
Verstecke lieber Frau und Tochter \\
denn ich bin der Schlage-Puchta. \\ \\

Ich bin neu hier, sei verflucht, \\
Der mich zu unterschätzen sucht. \\
Ich mag Fischer, ich mag BLYM, \\
 da ich der Herr Leck ja bin. \\ \\

Ich lieb' Geos, welch ein Glück, \\
schreck' nicht vor 3D zurück. \\
Mach' Seminare schon seit jeher, \\
und ich heiß' Professor Dreher. \\ \\

Geos mag ich auch, doch nicht wie du denkst, \\
rechne es doch mal komplex! \\
its-shirts sind ja der Brüller, \\
finde ich, Herr Doktor Müller. \\ \\

Nun sind wir unter einem Dach, \\
Warnemünde bis Oberwolfach. \\
Und fühlt euch nicht gleich überrollt, \\
falls ihr zu der IMO wollt.
\onehalfspacing
\chapter{Hinführung} %bis 2.1
\begin{chapquote}{Patrick Nasri-Roudsari}
\glqq AIMO sind objektiv massig Autisten.\grqq
\end{chapquote}
     \textit{Der 9. Mai 2020} \\
\chapter{Motivation} %bis 2.2
\begin{chapquote}{Veran Stojanović}
\glqq Abends passieren die wichtigen Dinge.\grqq
\end{chapquote}
%2.2

\chapter{Beweis} %bis 2.6
\begin{chapquote}{Sören Kierkegaard}
\glqq Oder ist etwa Schwermut nicht das Gebrechen der Zeit, ist sie es nicht, die selbst in deren leichtsinnigem Gelächter widerhallt?\grqq
\end{chapquote}
     %2.3
     \textit{Der 10. Mai 2020} \\
     
     \bigskip \bigskip \bigskip \bigskip
     %2.4
     
     \bigskip \bigskip \bigskip \bigskip
     %2.5
     
     
     \bigskip \bigskip \bigskip \bigskip
     %2.6

\chapter{Rückbezug} %bis 3.1
\begin{chapquote}{Patrick Nasri-Roudsari}
\glqq Mal sehen schreib ich die Fanfiction mit: Hat Jesus die Bibel geschrieben?\grqq
\end{chapquote}

\chapter{Abschließende Überlegungen} %bis 4.2
\begin{chapquote}{Georgi Kocharyan}
\glqq Es muss tierisch sinnlich sein\grqq
\end{chapquote}


  \bigskip \bigskip \bigskip \bigskip
     %4.2
     
     
\chapter{Im Kerker} %bis 4.3
\begin{chapquote}{Paul Pelisson}
\glqq Grandeur, savoir, renommée, \\
Amitié, plaisir et bien, \\
Tout n’est que vent, que fumée, \\
Pour mieux dire, tout n’est rien. 
\grqq
\end{chapquote}
\end{document}
